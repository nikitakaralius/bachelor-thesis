\chapter*{Введение}
\addcontentsline{toc}{chapter}{Введение}

\textbf{Актуальность исследования.}
Современное развитие ИТ сопровождается широким переходом к распределённым системам и микросервисной архитектуре, обеспечивающим гибкость и масштабируемость приложений.
Вместе с тем раздробленность системы увеличивает поверхность атаки: каждое микросервисное приложение может стать отдельной точкой уязвимости.
В таких условиях повышается значение межсервисной безопасности, включая надёжную аутентификацию и авторизацию.
Без эффективного управления секретами (ключами, токенами) распределённые системы становятся уязвимыми к компрометации.
Для защиты взаимодействия между сервисами и пользователя используются современные протоколы аутентификации и авторизации (OAuth 2.0, OpenIDConnect, JWT, mTLS и др.), которые гарантируют, что к ресурсам системы получают доступ только уполномоченные субъекты.
Одновременно растёт потребность в системном аудите: сбор и анализ логов операций в распределённых приложениях обеспечивает подотчётность и помогает выявлять инциденты безопасности.
Так, в микросервисных средах журналирование является фундаментальным для обеспечения принципов трассируемости и обнаружения аномалий.
Например, в Kubernetes функция аудита формирует хронологический набор записей о действиях в кластере, что подчёркивает роль централизованного логирования.
Таким образом, совокупность нерешённых проблем межсервисной аутентификации, управления доступом и учёта событий в распределённых системах обуславливает актуальность данного исследования.


\textbf{Предметная область.}
В рамках работы изучаются вопросы безопасного взаимодействия микросервисов и управления доступом в Kubernetes-кластерах, а также организации аутентификации и авторизации пользователей через внешние и внутренние провайдеры.
Особое внимание уделяется архитектурным подходам (mTLS-прокси, API Gateway, SSO) и протоколам безопасности (OAuth 2.0, OpenID Connect, JWT, RBAC/ABAC), а также практическим механизмам сбора и анализа событий.
Изучение данной предметной области основано на анализе теоретических источников и обзора практических решений в системе с открытым исходным кодом.

\textbf{Практическая значимость.}
Результаты исследования и разработанный проект системы аутентификации, авторизации и аудита в микросервисной среде имеют прикладную ценность:

\begin{itemize}
  \item Реализованное ПО на базе современных протоколов (mTLS, OAuth 2.0, OpenID Connect, JWT и др.) позволяет решать практические задачи обеспечения безопасности распределённых приложений, гарантируя доверенное взаимодействие сервисов и контроль прав пользователей.
  \item Предложенные подходы и архитектурные решения могут стать основой для дальнейших научных исследований и развития систем безопасности микроcервисов, а также для внедрения аналогичных решений в индустрии.
  \item Разработанный проект может быть применён организациями при построении процессов CI/CD и мониторинга микросервисов, что повысит надёжность и удобство аудита операций, облегчая соответствие нормативам и ускоряя обнаружение инцидентов.
\end{itemize}

