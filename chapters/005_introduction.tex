\chapter*{Введение}
\addcontentsline{toc}{chapter}{Введение}

Современные программные системы все чаще используют микросервисную архитектуру и облачную оркестрацию~\cite{Stojanov2024}.
Этот подход позволяет эффективно масштабировать и независимо развивать компоненты,
однако одновременно увеличивает сложность управления безопасностью:
дробление приложения на множество сервисов расширяет поверхность атаки и
усложняет обеспечение корректной идентификации и разграничения прав как для пользователей,
так и для самих сервисов~\cite{Glumov2024}.
В~распределенных средах особенно актуальны вопросы надежной пользовательской аутентификации и авторизации,
безопасной межсервисной коммуникации и организации централизованного аудита для обнаружения инцидентов~\cite{Zimina2023, Barabanov2021}.
Низкая степень унификации и ограниченная доступность готовых решений с открытым исходным кодом,
полностью покрывающих требования по AAA (Authentication, Authorization, Accounting),
создают потребность в исследованиях и разработке собственных решений,
пригодных для промышленного использования~\cite{Almeida2022}.

\textbf{Объектом исследования} является процесс обеспечения безопасности в распределенной
микросервисной платформе непрерывной доставки, функционирующей в среде Kubernetes.

\textbf{Предметом исследования} выступает AAA-подсистема указанной платформы.
Данная подсистема разрабатывается в рамках настоящей работы как составная часть более широкого проекта,
реализуемого коллективом разработчиков.
В~то время как другие участники проекта занимаются проектированием и реализацией специализированных компонентов платформы
(таких как управление релизами, доставка на сервер, хранение артефактов),
предлагаемая AAA-подсистема предоставляет единые механизмы безопасности,
которыми все компоненты платформы будут пользоваться для обеспечения доверенного взаимодействия.

\textbf{Целью работы} является разработка и реализация компонентов подсистемы аутентификации,
авторизации и учета для распределенной микросервисной платформы непрерывной доставки,
обеспечивающих безопасное взаимодействие пользователей и сервисов.

Для достижения поставленной цели необходимо решить следующие \textbf{задачи}:
\begin{enumerate}
  \item Изучить современные подходы и механизмы обеспечения безопасности в распределенных микросервисных системах.
  \item Спроектировать архитектуру AAA-подсистемы, включающую компоненты пользовательской и межсервисной аутентификации и авторизации,
        а также централизованного аудита.
  \item Реализовать компоненты пользовательской аутентификации и авторизации.
  \item Реализовать компоненты межсервисной аутентификации и авторизации.
  \item Реализовать компоненты аудита и мониторинга.
  \item Провести апробацию разработанной AAA-подсистемы.
\end{enumerate}

В~основу работы положена \textbf{гипотеза} о том, что для обеспечения комплексной безопасности распределенной
микросервисной платформы в среде Kubernetes оптимальным является комбинированный подход,
сочетающий специализированные инструменты для различных аспектов AAA\@.
В частности, предполагается следующее:
\begin{itemize}
  \item Применение Service Mesh с автоматическим управлением сертификатами и взаимной TLS-аутентификацией
        позволит обеспечить надежную межсервисную коммуникацию без необходимости ручной настройки защищенных каналов для каждого взаимодействия.
  \item Использование готовых сторонних провайдеров идентификации для централизованного управления пользователями
        и их правами в системе с интеграцией через стандартные протоколы OAuth~2.0/OIDC обеспечит
        гибкую и масштабируемую пользовательскую аутентификацию и авторизацию.
  \item Архитектура централизованного логирования и аудита с использованием sidecar-агентов для сбора структурированных логов,
        включающих корреляционные идентификаторы и категоризацию событий,
        обеспечит эффективное обнаружение инцидентов безопасности и упростит анализ распределенных транзакций.
\end{itemize}

\textbf{Практическая значимость} работы определяется несколькими факторами.
Во-первых, платформы непрерывной доставки, как правило, получают доступ с высокими привилегиями к критической инфраструктуре организации.
Неспособность такой платформы обеспечить должный уровень безопасности делает ее неконкурентоспособной и непригодной для промышленной эксплуатации~\cite{NIST2023}.
Во-вторых, разработанная AAA-подсистема представляет собой прикладную реализацию рекомендаций по безопасной межсервисной аутентификации и управлению доступом,
предоставляя команде платформы готовый модуль для обеспечения доверенного взаимодействия ее компонентов.
Сформулированные архитектурные решения и практические рекомендации могут служить основой для дальнейших исследований и эволюции системы,
в~том числе для расширения поддержки внешних провайдеров аутентификации, уточнения политик доступа и интеграции с системами аналитики и безопасности.
Наконец, разработанная подсистема ориентирована на решение конкретных практических задач безопасности и аудита
в~пределах микросервисной экосистемы платформы непрерывной доставки,
будучи естественно интегрированной в ее архитектуру при сохранении возможности расширения в будущем.
