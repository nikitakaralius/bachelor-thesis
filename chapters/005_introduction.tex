\chapter*{Введение}
\addcontentsline{toc}{chapter}{Введение}

\textbf{Актуальность исследования}
Современные программные системы всё чаще используют микросервисную архитектуру и облачную оркестрацию. %[ссылка]
Это подход даёт преимущества в масштабируемости и независимом развитии компонентов, но одновременно увеличивает сложность управления безопасностью:
дробление приложения на множество компонентов расширяет поверхность атаки и усложняет гарантирование корректной идентификации и разграничения прав как для пользователей, так и для самих сервисов.
В распределённых средах особенно актуальны вопросы надёжной пользовательской аутентификации и авторизации, безопасной межсервисной коммуникации и организации централизованного аудита и логирования для обнаружения инцидентов.
Низкая степень унификации и ограниченная доступность готовых Open Source решений, полностью покрывающих требования по AAA (Authentication, Authorization, Accounting) создаёт потребность в исследованиях и разработке собственных решений, пригодных для промышленного использования.
Эти факторы обусловливают актуальность и необходимость исследования по проектированию и реализации AAA компонентов для распределённых микросервисных систем ~\cite{Almeida2022, Glumov2024, Zimina2023, KubernetesAudit2025}.

\textbf{Предметная область}
Предметом исследования является проектирование и реализация AAA подсистемы для распределённой микросервисной Continuous Delivery платформы с открытым исходным кодом, развёрнутой в Kubernetes.
В рамках предметной области изучаются:
\begin{itemize}
  \item Архитектурные подходы обеспечения доверенной межсервисной коммуникации;
  \item Протоколы и механизмы пользовательской аутентификации и авторизации;
  \item Архитектура и практики сбора, передачи и индексирования продуктовых логов и событий аудита;
\end{itemize}
Особая практическая задача — обеспечить, чтобы разработанная AAA-подсистема была естественно интегрирована в архитектуру Continuous Delivery платформы,
и обслуживала исключительно микросервисы и потоки данных внутри данной системы при сохранении возможности расширения в будущем ~\cite{DiPilla2023}.

\textbf{Практическая значимость} выполненной разработки заключается в следующем:
\begin{itemize}
  \item Разработанный проект AAA станет прикладной реализацией рекомендаций по безопасной межсервисной аутентификации и управлению доступом.
  Это даст команде платформы готовое решение для обеспечения доверенного взаимодействия компонентов платформы.
  \item Архитектурные решения и практические рекомендации, сформированные в работе, могут быть использованы для дальнейших исследований и эволюции системы.
  \item Разрабатываемая подсистема обеспечит выполнение практических задач безопасности и аудита.
\end{itemize}

\cite{Barabanov2021, Glumov2024, Sikha2025}.

