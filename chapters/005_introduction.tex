\chapter*{Введение}
\addcontentsline{toc}{chapter}{Введение}

Современные программные системы всё чаще используют микросервисную архитектуру и облачную оркестрацию~\cite{Stojanov2024}.
Этот подход позволяет эффективно масштабировать и независимо развивать компоненты,
однако одновременно увеличивает сложность управления безопасностью:
дробление приложения на множество сервисов расширяет поверхность атаки и
усложняет обеспечение корректной идентификации и разграничения прав как для пользователей,
так и для самих сервисов~\cite{Glumov2024}.
В~распределённых средах особенно актуальны вопросы надёжной пользовательской аутентификации и авторизации,
безопасной межсервисной коммуникации и организации централизованного аудита для обнаружения инцидентов~\cite{Zimina2023, Barabanov2021}.
Низкая степень унификации и ограниченная доступность готовых решений с открытым исходным кодом,
полностью покрывающих требования по AAA (Authentication, Authorization, Accounting),
создаёт потребность в исследованиях и разработке собственных решений,
пригодных для промышленного использования~\cite{Almeida2022}.

\textbf{Целью работы} является разработка и реализация компонентов подсистемы аутентификации, авторизации и учёта для распределённой микросервисной платформы непрерывной доставки, обеспечивающих безопасное взаимодействие пользователей и сервисов.

Для достижения поставленной цели необходимо решить следующие \textbf{задачи}:
\begin{enumerate}
  \item Изучить современные подходы и механизмы обеспечения безопасности в распределённых микросервисных системах.
  \item Спроектировать архитектуру AAA-подсистемы, включающую компоненты пользовательской и межсервисной аутентификации и авторизации, а также централизованного аудита.
  \item Реализовать компоненты пользовательской аутентификации и авторизации.
  \item Реализовать компоненты межсервисной аутентификации и авторизации.
  \item Реализовать компоненты аудита и мониторинга.
  \item Провести апробацию разработанной AAA-подсистемы.
\end{enumerate}

\textbf{Предметом исследования} является AAA-подсистема для распределённой микросервисной платформы непрерывной доставки (Continuous Delivery), функционирующей в среде Kubernetes.
В~рамках предметной области изучаются архитектурные подходы к обеспечению доверенной межсервисной коммуникации,
протоколы и механизмы пользовательской аутентификации и авторизации,
а~также архитектура и практики сбора, передачи и индексирования продуктовых событий.

\textbf{Практическая значимость} работы определяется несколькими факторами.
Во-первых, платформы непрерывной доставки, как правило, получают доступ с высокими привилегиями к критической инфраструктуре организации.
Неспособность такой платформы обеспечить должный уровень безопасности делает её неконкурентоспособной и непригодной для промышленной эксплуатации~\cite{NIST2023}.
Во-вторых, разработанный проект AAA представляет собой прикладную реализацию рекомендаций по безопасной межсервисной аутентификации и управлению доступом,
предоставляя команде платформы готовый модуль для обеспечения доверенного взаимодействия её компонентов.
Сформулированные архитектурные решения и практические рекомендации могут служить основой для дальнейших исследований и эволюции системы,
в~том числе для расширения поддержки внешних провайдеров аутентификации, уточнения политик доступа и интеграции с системами аналитики и безопасности.
Наконец, разработанная подсистема ориентирована на решение конкретных практических задач безопасности и аудита
в~пределах микросервисной экосистемы платформы непрерывной доставки,
будучи естественно интегрированной в её архитектуру при сохранении возможности расширения в будущем.
