\chapter*{Введение}
\addcontentsline{toc}{chapter}{Введение}

% Актуальность
% Вот это важный пункт в актуальность – Платформа будет иметь доступ с высокими привилегиями к критической инфраструктуре клиента. Если платформа не может обеспечить безопасность, то такая платформа неконкурентоспособна.
% Но его надо как-то интегрировать с тем, что идет в след абзаце. Без этого контекста будет странно.
Современные программные системы всё чаще используют микросервисную архитектуру и облачную оркестрацию~\cite{Stojanov2024}.
Это подход даёт преимущества в масштабируемости и независимом развитии компонентов,
но одновременно увеличивает сложность управления безопасностью:
дробление приложения на множество компонентов расширяет поверхность атаки и
усложняет гарантирование корректной идентификации и разграничения прав как для пользователей,
так и для самих сервисов~\cite{Glumov2024}.
В~распределённых средах особенно актуальны вопросы надёжной пользовательской аутентификации и авторизации,
безопасной межсервисной коммуникации и организации централизованного аудита и логирования для обнаружения инцидентов~\cite{Zimina2023, Barabanov2021}.
Низкая степень унификации и ограниченная доступность готовых Open Source решений,
полностью покрывающих требования по AAA (Authentication, Authorization, Accounting),
создаёт потребность в исследованиях и разработке собственных решений,
пригодных для промышленного использования~\cite{Almeida2022}. % спорно сюда этот источник вставлять

% Где цель? Из слайдов – Разработка и реализация компонентов подсистемы аутентификации, авторизации и учета (далее AAA) для распределенной микросервисной платформы, обеспечивающей безопасное взаимодействие пользователей и сервисов.
% Но это скам какой-то. Буквально дублирует тему.
% Задачи сформулировать списком, чтобы явно их можно было выделить
% Предмет исследования + задачи. Задачи пересмотреть, референс – декабрьские слайды.
Предметом исследования данной работы является проектирование и реализация AAA подсистемы
для распределённой микросервисной платформы непрерывной доставки (Continuous Delivery), работающей в Kubernetes.
В рамках предметной области изучаются архитектурные подходы обеспечения доверенной межсервисной коммуникации,
протоколы и механизмы пользовательской аутентификации и авторизации,
архитектура и практики сбора, передачи и индексирования продуктовых логов и событий аудита.
Особая практическая задача — обеспечить, чтобы разработанная AAA-подсистема была естественно интегрирована в архитектуру Continuous Delivery платформы,
и обслуживала исключительно микросервисы и потоки данных внутри данной системы при сохранении возможности расширения в будущем.

% Практическая значимость
Разработанный проект AAA представляет собой прикладную реализацию рекомендаций по безопасной межсервисной аутентификации и управлению доступом,
предоставляя команде платформы готовый модуль для обеспечения доверенного взаимодействия её компонентов.
Сформулированные архитектурные решения и практические рекомендации могут служить основой для дальнейших исследований и эволюции системы,
в том числе для расширения поддержки внешних провайдеров аутентификации, уточнения политик доступа и интеграции с системами аналитики и безопасности.
Наконец, разработанная подсистема обеспечит решение практических задач безопасности и аудита
в пределах микросервисной экосистемы разрабатываемой платформы непрерывной доставки,
будучи ориентированной на внутренние процессы проекта и его инфраструктуру.
