\chapter{Анализ подходов к построению подсистем AAA в микросервисных архитектурах} \label{ch:1}

% TODO: change this part according to sections
Хорошим стилем является наличие введения к главе, которое \textit{начинается непосредственно после названия главы, без оформления в виде отдельного параграфа}.
Во введении может быть описана цель написания главы, а также приведена краткая структура главы.
Например, в параграфе~\ref{sec:1.1} приведены примеры оформления одиночных формул, рисунков и таблицы.
Параграф~\ref{sec:1.2} посвящён многострочным формулам и сложносоставным рисункам и т.д.

\section{Обзор имеющихся решений и исследований в указанной области} \label{sec:1.1}

\subsection{Пользовательская аутентификация и авторизация} \label{subsec:1.1.2}
% Начинать надо с того, что такое пользовательская аутентификация.
% Здесь можно затронуть разные модели, которые изучали на курсе ИБ: мандатная, модель Биба, RBAC.
% Вырулить что RBAC выглядит для нас подходящей.
% Здесь можно начать JWT, oauth, OIDC.
% Но это всего лишь протоколы, а нам нужны их реализации для использования.
% Нужно прийти к использованию IdP и обосновать почему не изобретаем свой велосипед.

\subsection{Межсервисная аутентификация и авторизация} \label{subsec:1.1.1}
% Первым делом, объяснить какую проблему решаем.
% Zero Trust, безопасность кластера в первую очередь.
% Про это прям источник был хороший.
% Источники, рассказывающие про привычные протоколы (JWT, oauth еще что-нибудь).
% Постепенно идти к TLS, mTLS, вишенка на торте – Service Mesh.
% Про TLS и mTLS можно остановиться подробнее.
% Выбор Service Mesh должен быть обусловлен сложностью ручной работы с сертификатами для mTLS.
% Про Service Mesh тоже подробно рассказать, что это за технология и что она дает.
% Для объяснения использовать много диаграмм, благо они есть.
% Вспомнить про предыдущую секцию. Что с запросами в кластере ходят не только микросервисы, но и пользователи.
% Эти два мира нужно между собой подружить – рассказать, что Service Mesh умеет такое.

\subsection{Логирование и аудит} \label{subsec:1.1.3}
% Тоже начать с того, какую проблему решаем (опираться на источники как и ранее).
% Рассказать про различные паттерны сбора логов (где-то было про это).
% Круто зайдут диаграммы. Особенно если рассказывать про HA кластер.
% Где-то было про запись в std output (подкрепить источник).
% Хотелось бы источник про продуктовые события, что это отдельный сервис, как с ним работать.
% Уделить внимание ресурсам, времени жизни – логов много, это все выливается в деньги.

\section{Выделение критериев сравнения и подбор инструментальных средств на их основе} \label{sec:1.2}

\section{Формулирование задачи и гипотезы ее решения} \label{sec:1.3}

\section{Выводы по главе \thechapter} \label{sec:1.4}
