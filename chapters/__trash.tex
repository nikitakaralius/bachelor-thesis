% В этом файле содержится текст, который сначала был добавлен, а затем сокращен по какой-либо причине.
% Зачастую, эта причина – компактность. Есть информация, которая просто лишняя и ее можно описать, сославшись на источник.
% Да, есть git вся фигня, но искать по истории, когда что убрано было – впадлу.

%% 006_chapter1 – Пользовательская аутентификация и авторизация
Современные системы используют различные модели контроля доступа,
выбор которых зависит от требований к безопасности и специфики предметной области~\cite{Stojanov2024}:
\begin{enumerate}
  \item \textbf{Дискреционный контроль доступа} (Discretionary Access Control, DAC) — модель,
  в которой владелец ресурса самостоятельно определяет права доступа к нему.
  Данная модель отличается простотой реализации, но имеет существенный недостаток:
  сложность централизованного управления политиками безопасности~\cite{NIST2023}.

  \item \textbf{Мандатный контроль доступа} (Mandatory Access Control, MAC) — модель,
  при которой права доступа определяются системой на основе уровней конфиденциальности субъектов и объектов.
  \item MAC обеспечивает высокий уровень безопасности и часто применяется в государственных и военных системах,
  однако отличается низкой гибкостью~\cite{NIST2023}.

  \item \textbf{Ролевой контроль доступа} (Role-Based Access Control, RBAC) — одна из наиболее распространенных моделей,
  в которой права доступа назначаются ролям, а пользователи получают доступ через членство в этих ролях~\cite{Glumov2024}.
  RBAC хорошо подходит для организаций с устоявшейся структурой и четким разделением обязанностей.

  \item \textbf{Атрибутный контроль доступа} (Attribute-Based Access Control, ABAC) — наиболее гибкая модель,
  в которой решения о доступе принимаются на основе атрибутов субъекта, объекта, действия и окружения~\cite{NIST2023}.
  ABAC позволяет реализовывать сложные политики доступа, учитывающие контекст запроса, что особенно актуально для микросервисных архитектур.
\end{enumerate}
%% end of 006_chapter1 – Пользовательская аутентификация и авторизация
