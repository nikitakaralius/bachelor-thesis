%%%% Шаблон ВКР <<SPbPU-student-thesis-template>>  %%%%
%%
%%   Создан на основе глубокой переработки шаблона российских кандидатских и докторских диссертаций [1]. 
%%   
%%   Полный список различий может быть получен командами git.
%%   Лист авторов-составителей расположен в README.md файле.
%%   Подробные инструкции по использованию в [1,2].
%%   
%%   Рекомендуем установить TeX Live + TeXstudio
%%   <<Стандартная>> компиляция 2-3 РАЗА с помощью pdflatex + biber (для библиографии)     
%%  
%%%% Student thesis template <<SPbPU-student-thesis-template>> %%%%
%%
%%   Created on the basis of deepl modifification of the Russian candidate and doctorate thesis template [1]. 
%%   
%%   Full list of differences can be achieved by git commands.
%%   List of template authors can be seen in the README.md file.
%%   Detailed instructions of usage, see, please in [1,2].
%%     
%%   [1] github.com/AndreyAkinshin/Russian-Phd-LaTeX-Dissertation-Template 
%%   [2] Author_guide_SPBPU-student-thesis-template.pdf
%%   
%%   It is recommended to install TeX Live + TeXstudio   
%%   Default compilation 2-3 TIMES with pdflatex + biber (for the bibliography)
%%  
%%%% Preamble start %%%%  
%%
%%   Please, do not modify files in the preamble
%%
\newcommand*{\anyptfilebase}{settings/bpfont} 
\newcommand*{\anyptsize}{14} 		 
\RequirePackage[l2tabu,orthodox]{nag} 
\documentclass[extrafontsizes,a4paper,*pt,oneside,openany]{memoir}
\input{settings/common/setup}               
\input{settings/common/packages}  
\input{settings/Dissertation/dispackages}         
\input{settings/Dissertation/userpackages}         
\input{settings/Dissertation/setup}               
\input{settings/Dissertation/preamblenames}       
\input{settings/common/styles}    
\input{settings/Dissertation/disstyles}           
\input{settings/Dissertation/userstyles}          
\input{settings/biblio/bibliopreamble}
\input{settings/Dissertation/inclusioncontrol}
\input{settings/common/TO-DO-list}
%%
%%%% Preamble end %%%%  % лучше не редактировать / please, keep unmodified

\setcounter{docType}{1} % лучше не редактировать / please, keep unmodified

%%%% Настройки автора / Author settings
%% 
\input{chapters/my_settings} % добавляем свои команды / update your commands

\begin{document} % начало документа


\input{settings/common/renames} % Заполнить сведения, 
										 % в т.ч. ключевые слова и аннотацию.

\input{chapters/title}					 % Титульный лист
										 % Убираем footnotes, консультанта, если нет

\input{chapters/task}					 % Задание 
										 % Для сдачи в высшую школу компилируем двухсторонний My_task.tex 
										 % После подписания задания изменение его содержания и оформления запрещено

\input{chapters/summary}			 	 % Реферат 
										 % Убираем footnotes, дубли команд \abstractEn и \abstractRu 
										

\input{chapters/contents}  	         % Оглавление


\input{chapters/introduction}	    	 % Введение

%% Начало основной части
\input{chapters/chapter1}	         	 % Глава 1
\ContinueChapterBegin % размещать главы <<подряд>> 
\input{chapters/chapter2}	         	 % Глава 2
\input{chapters/chapter3}           	 % Глава 3
\input{chapters/chapter4}           	 % Глава 3
\ContinueChapterEnd % завершить размещение глав <<подряд>>
%% Завершение основной части

\input{chapters/conclusion}        	 % Заключение

%% Наличие следующих перечней не исключает расшифровку сокращения и условного обозначения при первом упоминании в тексте!
\input{chapters/acronyms}		         % Необязательная рубрика! Список сокращений и условных обозначений

\input{chapters/dictionary}    		 % Необязательная рубрика! Словарь терминов
% По порядку после Списка сокращений и условных обозначений, если есть.	


\input{chapters/references}		     % Список литературы

% Здесь можно поместить список иллюстративного материала

\appendix % не редактировать / keep unmodified


\input{chapters/appendix1}			     % Приложение 1

\input{chapters/appendix2}			 	 % Приложение 2


\end{document} % конец документа


%%% Удачной защиты ВКР! - Good luck on the thesis defense!
%%
%%% Поддержать проект
%%
%% Запросы на добавление / изменение просим писать на следующей странице:
%% https://github.com/ParkhomenkoV/SPbPU-student-thesis-template/issues
%%
%% Список пожеланий в файле шаблона <<TO-DO-list.tex>>
%%
%% Благодарности просим указывать в виде 
%%
%% 1. Добавление <<Звезды>> проекту https://github.com/ParkhomenkoV/SPbPU-student-thesis-template/stargazers
%%
%% 2. Добавления <<Сердечка>> и репоста проекта в социальных сетях:
%%		https://vk.com/latex_polytech 
%%		https://www.fb.com/groups/latex.polytech
%%

%%% Support project
%%
%% Requests on adding / modifications is better to be publishen on the following web-page:
%% https://github.com/ParkhomenkoV/SPbPU-student-thesis-template/issues
%%
%% Wishlist is in the template's file called <<TO-DO-list.tex>>
%%
%% Acknowledgements are better to be done in the form of 
%%
%% 1. Adding <<Star>> to the project https://github.com/ParkhomenkoV/SPbPU-student-thesis-template/stargazers
%%
%% 2. Adding <<Likes>> and Project repost in the social networks:
%%		https://vk.com/latex_polytech 
%%		https://www.fb.com/groups/latex.polytech
%% 

% Check list при передаче ВКР:
% - Зачистка всех вспомогательных файлов (Clear auxilary files) и компиляция ВКР не менее 3х раз