%%%% Preamble start %%%%  
%%
%%   Please, do not modify files in the preamble
%%
\newcommand*{\anyptfilebase}{settings/bpfont} 
\newcommand*{\anyptsize}{14} 		 
\RequirePackage[l2tabu,orthodox]{nag} 
\documentclass[extrafontsizes,a4paper,*pt,oneside,openany]{memoir}
\input{settings/common/setup}               
\input{settings/common/packages}  
\input{settings/Dissertation/dispackages}         
\input{settings/Dissertation/userpackages}         
\input{settings/Dissertation/setup}               
\input{settings/Dissertation/preamblenames}       
\input{settings/common/styles}    
\input{settings/Dissertation/disstyles}           
\input{settings/Dissertation/userstyles}          
\input{settings/biblio/bibliopreamble}
\input{settings/Dissertation/inclusioncontrol}
\input{settings/common/TO-DO-list}
%%
%%%% Preamble end %%%% 
\setcounter{docType}{1}
\input{chapters/common_settings}

\begin{document}
  \input{template_settings/common/renames}

  % Титульник
  \input{chapters/000_nrw_title}

  % Для сдачи в высшую школу компилируем двухсторонний My_task.tex
  % После подписания задания изменение его содержания и оформления запрещено
  % \input{chapters/002_task}

  % Реферат
  % Убираем footnotes, дубли команд \abstractEn и \abstractRu
  % \input{chapters/003_summary}

  % Оглавление
  \setlength{\parskip}{0.35\onelineskip} % интервал между элементов - полуторный
\begin{Spacing}{\Single}               % интервал внутри элемента - одинарный
  \tableofcontents
\end{Spacing}
\setlength{\parskip}{0pt}              % интервал между элементов - полуторный
\OnehalfSpacing*                       % Полуторный интервал % * to force it in the floats
\newpage


  % Введение
  \chapter*{Введение}
\addcontentsline{toc}{chapter}{Введение}

Современные программные системы всё чаще используют микросервисную архитектуру и облачную оркестрацию~\cite{Stojanov2024}.
Этот подход позволяет эффективно масштабировать и независимо развивать компоненты,
однако одновременно увеличивает сложность управления безопасностью:
дробление приложения на множество сервисов расширяет поверхность атаки и
усложняет обеспечение корректной идентификации и разграничения прав как для пользователей,
так и для самих сервисов~\cite{Glumov2024}.
В~распределённых средах особенно актуальны вопросы надёжной пользовательской аутентификации и авторизации,
безопасной межсервисной коммуникации и организации централизованного аудита для обнаружения инцидентов~\cite{Zimina2023, Barabanov2021}.
Низкая степень унификации и ограниченная доступность готовых решений с открытым исходным кодом,
полностью покрывающих требования по AAA (Authentication, Authorization, Accounting),
создаёт потребность в исследованиях и разработке собственных решений,
пригодных для промышленного использования~\cite{Almeida2022}.

\textbf{Объектом исследования} является процесс обеспечения безопасности в распределённой микросервисной платформе непрерывной доставки, функционирующей в среде Kubernetes.

\textbf{Предметом исследования} выступает AAA-подсистема указанной платформы.
Данная подсистема разрабатывается в рамках настоящей работы как составная часть более широкого проекта,
реализуемого коллективом разработчиков.
В~то время как другие участники проекта занимаются проектированием и реализацией специализированных компонентов платформы
(таких как управление релизами, доставка на сервер, хранение артефактов),
предлагаемая AAA-подсистема предоставляет единые механизмы безопасности,
которыми все компоненты платформы будут пользоваться для обеспечения доверенного взаимодействия.

\textbf{Целью работы} является разработка и реализация компонентов подсистемы аутентификации, авторизации и учёта для распределённой микросервисной платформы непрерывной доставки, обеспечивающих безопасное взаимодействие пользователей и сервисов.

Для достижения поставленной цели необходимо решить следующие \textbf{задачи}:
\begin{enumerate}
  \item Изучить современные подходы и механизмы обеспечения безопасности в распределённых микросервисных системах.
  \item Спроектировать архитектуру AAA-подсистемы, включающую компоненты пользовательской и межсервисной аутентификации и авторизации, а также централизованного аудита.
  \item Реализовать компоненты пользовательской аутентификации и авторизации.
  \item Реализовать компоненты межсервисной аутентификации и авторизации.
  \item Реализовать компоненты аудита и мониторинга.
  \item Провести апробацию разработанной AAA-подсистемы.
\end{enumerate}

В~основу работы положена \textbf{гипотеза} о том, что для обеспечения комплексной безопасности распределённой микросервисной платформы в среде Kubernetes оптимальным является комбинированный подход, сочетающий специализированные инструменты для различных аспектов AAA.
В~частности, предполагается следующее:
\begin{enumerate}
  \item Применение Service Mesh с автоматическим управлением сертификатами и взаимной TLS-аутентификацией позволит обеспечить надёжную межсервисную коммуникацию без необходимости ручной настройки защищённых каналов для каждого взаимодействия.
  \item Использование готовых сторонних провайдеров идентификации для централизованного управления пользователями и их правами в системе с интеграцией через стандартные протоколы OAuth~2.0/OIDC обеспечит гибкую и масштабируемую пользовательскую аутентификацию и авторизацию.
  \item Архитектура централизованного логирования и аудита с использованием sidecar-агентов для сбора структурированных логов, включающих корреляционные идентификаторы и категоризацию событий, обеспечит эффективное обнаружение инцидентов безопасности и упростит анализ распределённых транзакций.
\end{enumerate}

\textbf{Практическая значимость} работы определяется несколькими факторами.
Во-первых, платформы непрерывной доставки, как правило, получают доступ с высокими привилегиями к критической инфраструктуре организации.
Неспособность такой платформы обеспечить должный уровень безопасности делает её неконкурентоспособной и непригодной для промышленной эксплуатации~\cite{NIST2023}.
Во-вторых, разработанная AAA-подсистема представляет собой прикладную реализацию рекомендаций по безопасной межсервисной аутентификации и управлению доступом,
предоставляя команде платформы готовый модуль для обеспечения доверенного взаимодействия её компонентов.
Сформулированные архитектурные решения и практические рекомендации могут служить основой для дальнейших исследований и эволюции системы,
в~том числе для расширения поддержки внешних провайдеров аутентификации, уточнения политик доступа и интеграции с системами аналитики и безопасности.
Наконец, разработанная подсистема ориентирована на решение конкретных практических задач безопасности и аудита
в~пределах микросервисной экосистемы платформы непрерывной доставки,
будучи естественно интегрированной в её архитектуру при сохранении возможности расширения в будущем.


  %% Начало основной части
  \chapter{Анализ подходов к построению подсистем AAA в микросервисных архитектурах} \label{ch:1}

\section{Обзор имеющихся решений и исследований в указанной области} \label{sec:1.1}

\section{Выделение критериев сравнения и подбор инструментальных средств на их основе} \label{sec:1.2}

\section{Формулирование задачи и гипотезы ее решения} \label{sec:1.3}

\section{Выводы} \label{sec:1.4}
	               % Глава 1
  \ContinueChapterBegin
  % \input{chapters/007_chapter2}	             % Глава 2
  % \input{chapters/008_chapter3}              % Глава 3
  % \input{chapters/009_chapter4}              % Глава 3
  % \ContinueChapterEnd

  % \input{chapters/010_conclusion}        	 % Заключение

  %% Наличие следующих перечней не исключает расшифровку сокращения и условного обозначения при первом упоминании в тексте!
  % \input{chapters/011_acronyms}		         % Необязательная рубрика! Список сокращений и условных обозначений

  % \input{chapters/012_dictionary}    		 % Необязательная рубрика! Словарь терминов
  % По порядку после Списка сокращений и условных обозначений, если есть.


  \input{chapters/013_references}         % Список литературы

  % Здесь можно поместить список иллюстративного материала
  \appendix % не редактировать / keep unmodified


  % \chapter{Скриншот проверки в системе Антиплагиат}\label{ch:appendix1}

\begin{figure}[h]
  \centering
  \includegraphics[width=\textwidth]{diagrams/img/antiplagiat-report}
  \caption{Проверка отчета в системе <<Антиплагиат>>}
  \label{fig:antiplagiat-report}
\end{figure}
         % Приложение 1

  % \input{chapters/015_appendix2}         % Приложение 2


\end{document}



